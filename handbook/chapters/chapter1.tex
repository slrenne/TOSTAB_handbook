
\chapter{Diagnostic process}

The diagnosis of bone and soft tissue tumors relies heavily on clinical information. There are six key factors to consider when evaluating these patients: gender, age, timing, location, size, and depth. This chapter will discuss each of these factors in detail and their significance in the clinical-pathological evaluation.

\section{Clinical Information}
\subsection{Gender}
Gender plays an important role in the diagnosis of bone and soft tissue tumors. Some tumors are more prevalent in one gender or even exclusive to a specific gender. For instance, leiomyosarcoma is much more common in females, as it often originates from the smooth muscle, particularly in the uterus. On the other hand, dedifferentiated leiomyosarcoma is more frequently seen in males.

\subsection{Age}
Age is also highly suggestive of certain tumor types. Some tumors occur almost exclusively in specific age groups. For example, congenital or infantile fibrosarcoma typically affects infants, while undifferentiated pleomorphic sarcoma (UPS) is mostly seen in elderly patients. Additionally, several translocation-associated sarcomas are more common in children, adolescents, or young adults. Each histological subtype often correlates with a specific age range.

\subsection{Timing}
The timing of disease progression gives important clues regarding the nature of the tumor. Benign lesions tend to be present for long periods, growing very slowly. Sarcomas, however, typically develop over the course of months to years. Therefore, the duration of the symptoms can provide insight into the aggressiveness of the tumor.

\subsection{Location}
The location of the tumor is often characteristic of specific tumor types. Some tumors are confined to certain areas of the body. For example, gastrointestinal stromal tumors (GISTs) are found exclusively in the gastrointestinal tract. Similarly, dedifferentiated leiomyosarcoma frequently arises in the retroperitoneum, while solitary fibrous tumors often occur in the pelvis, retroperitoneum, or pleural space. Leiomyosarcoma may develop in areas associated with smooth muscle, such as cutaneous, vascular (e.g., saphenous vein), or uterine locations.

\subsection{Size}
Tumor size is another important factor. Masses larger than 5 cm are generally more concerning, and guidelines recommend a biopsy for any unexplained mass larger than 5 cm or any deep mass. 

\subsection{Depth}
The depth of a tumor is categorized as either superficial or deep, with superficial masses located above the fascia covering the muscle, while deep masses lie beneath it. Most deep tumors tend to be malignant, but there are exceptions. Dermatofibrosarcoma protuberans (DFSP), for instance, is a superficial tumor despite being a sarcoma. Other tumors, such as epithelioid sarcoma and myxofibrosarcoma, may also present as superficial masses. On the other hand, benign deep tumors, such as intramuscular myxoma and schwannoma, can mimic malignancy.

\section{Notable Exceptions}
\subsection{Superficial Sarcomas}
While most superficial tumors are benign, certain sarcomas such as angiosarcoma and Kaposi sarcoma can present superficially. Angiosarcoma may occur in the skin or breast, while Kaposi sarcoma is a distinctly superficial lesion.

\subsection{Benign Deep Tumors}
Despite the general rule, some deep-seated masses are benign. Intramuscular myxoma is a classic example, often found in females around 50 years old, typically within the thigh muscles (e.g., vastus medialis). Another example is intramuscular lipoma, which arises within muscle tissue rather than superficially, as is more common with lipomas.

\section{Simulators of Sarcoma}
\subsection{Nodular Fasciitis}
Nodular fasciitis is a benign lesion that can mimic sarcoma both clinically and histologically. It is associated with USP6 rearrangement and presents as a rapidly growing mass, usually smaller than 3 cm and almost always under 5 cm. Despite its rapid growth and concerning histologic features (e.g., mitotic activity, hemorrhage), its benign nature can often be confirmed by the small size and characteristic timing (developing over months).

\subsection{Other Sarcoma Simulators}
There are additional conditions that can clinically simulate sarcomas, which will be discussed further in subsequent sections.

\section{The diagnosis}

In diagnostic practice, we approach the evaluation of tissue in a systematic way, focusing on key elements such as clinical information (discussed in the previous chapter), morphological features (which will be the focus of this chapter), and mechanical features (to be covered in the next chapter). However, it is important to emphasize that the diagnostic process is not a rigid, step-by-step algorithm.

\section{A Non-Algorithmic Approach to Diagnosis}
The diagnostic process does not proceed in a strictly sequential manner, where one first examines the clinical information, then the morphology of the cells, followed by the tumor architecture, growth patterns, and immunohistochemistry, before finally arriving at a diagnosis. In practice, the evaluation often occurs in a more integrative and iterative way. 

\subsection{Dynamic Process of Diagnosis}
A pathologist may move back and forth between different aspects of the diagnostic process, constantly re-evaluating previous information as new insights emerge. For instance, clinical details might be reconsidered in light of a particular morphological feature, or immunohistochemical results might prompt a closer look at the cellular architecture. This iterative approach allows for a more flexible and nuanced diagnosis, where each piece of information can refine and adjust the diagnostic trajectory.

\subsection{Cognitive Recognition in Diagnosis}
Diagnosis is not primarily a process of exclusion, where one gradually narrows down the differential diagnosis. In the vast majority of cases, diagnosis occurs through recognition, a cognitive process akin to how one recognizes a familiar person.

Just as you recognize someone you know based on distinct features -- such as the shape of the nose, the sound of the voice, or the typical hairstyle -- diagnosis works similarly. For instance, if a familiar person suddenly changes their hairstyle, you would not be puzzled because other features (like their voice or facial structure) confirm their identity. The same holds true in pathology: you often recognize a disease based on key morphological and clinical clues, even if some features vary.

\subsection{Type 1 Thinking in Diagnosis}
This recognition process is what psychologist Daniel Kahneman refers to as ``Type 1 thinking'' -- fast, intuitive thinking based on experience. It contrasts with ``Type 2 thinking,'', which involves slower, more deliberate analysis. In pathology, Type 1 thinking is often the primary mechanism of diagnosis, as the pathologist recognizes disease entities immediately upon seeing certain patterns under the microscope, supported by clinical information.

While we teach diagnostic criteria in a systematic way, it is important for the reader to understand that these features are not always methodically checked off to arrive at a diagnosis. More often, the diagnosis emerges almost instantly based on familiarity with the entities in question.

\subsection{Mastering Diagnostic Entities}
To facilitate this cognitive recognition, the pathologist must be familiar with the specific entities within the menu of possible diagnoses. Just as the human brain can manage the recognition of hundreds of individuals, a trained pathologist can comfortably handle the recognition of around 100-150 distinct pathological entities. This manageable number of diseases is sufficient for most diagnostic practice in bone and soft tissue pathology, making it feasible for anyone to master.

\section{Overview of Morphological Features}
In this chapter, we will focus on the morphological characteristics that play a crucial role in the diagnosis of bone and soft tissue tumors. These features can be divided into several key categories, which will be explored in the following sections:
\begin{itemize}
    \item Cellular morphology
    \item Tissue architecture
    \item Growth patterns
    \item Tumor-stromal interaction
    \item Necrosis and other secondary changes
\end{itemize}

\section{Cellular Morphology}

The first morphological feature that is typically assessed in tumor diagnostics is the shape of the cells. Historically, this has been the foundation for the classification of mesenchymal tumors. Various cell shapes are recognized, each associated with different types of tumors.

\subsection{Spindle Cells}
Spindle cells are a common cell shape in mesenchymal tumors. They have an elongated shape with a long axis and two tapered ends, resembling a spindle. The classical example is \emph{fibromatosis}, where spindle cells predominate.

The nuclei in spindle cells can vary:
\begin{itemize}
    \item \textbf{Cigar-shaped nuclei}: These are oval, blunt-ended nuclei, typical of smooth muscle tumors, such as \emph{leiomyoma} or \emph{leiomyosarcoma}.
    \item \textbf{Spindled nuclei}: Seen in tumors of peripheral nerve origin, such as \emph{neurofibroma} or \emph{schwannoma}.
\end{itemize}

\subsection{Epithelioid Cells}
Epithelioid cells have nearly uniform dimensions in all three planes, giving them a more rounded or polygonal appearance in two-dimensional sections. Examples include \emph{epithelioid sarcoma} and \emph{alveolar soft part sarcoma}.

Variations in nuclear placement within these cells can occur:
\begin{itemize}
    \item \textbf{Centrally placed nucleus}: Often found in epithelioid cells.
    \item \textbf{Eccentric nucleus}: This gives the cell a \emph{rhabdoid} or \emph{plasmacytoid} appearance.
\end{itemize}

\subsection{Round Cells}
Round cells have less cytoplasm compared to the nucleus, making the cell appear small and round. The cytoplasm may sometimes be glycogen-rich, as seen in \emph{Ewing sarcoma}. These cells can appear almost devoid of cytoplasm under routine microscopy.

\textbf{Example: Ewing Sarcoma.} The morphology of round cells in \emph{Ewing sarcoma} shows scant cytoplasm, with predominantly nuclear content. Special stains, such as periodic acid-Schiff (PAS) for glycogen, can highlight the glycogen content, especially in frozen sections where processing is incomplete.

\subsection{Pleomorphic Cells}
Pleomorphic cells exhibit significant variation in size and shape of both the cells and their nuclei, often signaling a higher degree of atypia. \emph{Pleomorphism} is a feature typically associated with malignancy, though the degree of pleomorphism is not always indicative of the tumor's aggressiveness.

For instance:
\begin{itemize}
    \item \emph{Pleomorphic liposarcoma} may display marked nuclear pleomorphism, but this does not necessarily correlate with a highly aggressive clinical course.
    \item \emph{Pleomorphic hyalinizing angiectatic tumor} is another example where pleomorphism does not predict an aggressive outcome.
\end{itemize}

\subsection{Biphasic Neoplasms}
Biphasic tumors contain two distinct cell types or morphologies. A classical example is \emph{synovial sarcoma}, which shows both spindle cells and epithelial cells. Other examples include \emph{schwannomas} and \emph{malignant peripheral nerve sheath tumors (MPNST)}, which rarely can exhibit mixed cellular populations especially in syndromic settings.

\subsection{Fat Cells and Lipomatous Tumors}
Fat cells are easily recognized under the microscope due to their characteristic clear cytoplasm. This results from the loss of lipid during tissue processing, leaving optically clear spaces.

\textbf{Examples:}
\begin{itemize}
    \item **Lipomas**: Benign tumors composed of mature fat cells.
    \item **Liposarcoma**: A malignant tumor where fat cells may appear more atypical.
    \item **Brown fat**: Exhibits smaller fat vacuoles and abundant cytoplasm with rich mitochondria.
\end{itemize}

Additionally, entities like **paraffinomas** and **silicone granulomas**, which are associated with prior plastic surgeries, may mimic lipomatous tumors but are distinguished by the history and clinical context.

\section{Architectural Patterns}

Tumor architecture provides valuable diagnostic clues. Some of the major architectural patterns are:

\subsection{Fascicular Pattern}
In this pattern, tumor cells are arranged in linear bundles or fascicles. This pattern is characteristic of **smooth muscle tumors** (such as leiomyosarcomas) and **fibrosarcomatous tumors**, including the "herringbone" pattern seen in fibrosarcoma, where fascicles are arranged at sharp angles.

\subsection{Plexiform Pattern}
A **plexiform** architecture resembles a network of branching structures and is classically seen in **neurofibromas**, where the cells grow along and around nerves in a tangled, plexiform arrangement.

\subsection{Palisading Pattern}
In **palisading**, cells align in parallel arrays, often around a central core. This is seen in **schwannomas**, where the cells form **Verocay bodies**, but it can also be present in other tumors, such as basal cell carcinoma.

\subsection{Storiform Pattern}
The **storiform** or cartwheel pattern, in which spindle cells radiate from a central point, is seen in tumors like **dermatofibrosarcoma protuberans** (DFSP) and other fibrosarcomatous neoplasms.

\subsection{Lobulated Pattern}
This pattern is characterized by the growth of tumor cells in rounded lobules, as seen in **lipomas** and **pleomorphic adenomas**, where the neoplasm pushes the surrounding stroma.

\section{Stromal Features}

The stroma is the tissue that forms the tumor's environment, consisting of extracellular matrix components like collagen and mucin. Its appearance can be crucial in diagnosis.

\subsection{Myxoid Stroma}
A myxoid stroma contains mucinous material and is characteristic of certain neoplasms, such as **myxoid liposarcoma** and **myxofibrosarcoma**.

\subsection{Collagen Deposition}
Collagen deposition is seen in various tumors. For instance:
\begin{itemize}
    \item **Sclerosing fibrosarcoma**: Characterized by dense collagen.
    \item **Low-grade fibromyxoid sarcoma**: May contain giant collagen rosettes.
\end{itemize}

\subsection{Bone and Cartilage Formation}
Some tumors exhibit metaplastic bone or cartilage formation. For example, **ossifying fibromyxoid tumor** features bone formation within a soft tissue neoplasm, while **metaplastic bone** may form in benign conditions such as **lipoma**.

\section{Other Morphological Considerations}
The tumor's interaction with the surrounding tissue, especially immune cells and vasculature, also plays a diagnostic role:
\begin{itemize}
    \item **Inflammatory cells**: Some tumors, like **well-differentiated liposarcoma**, are associated with marked inflammation.
    \item **Vessel formation**: Abnormal vessels may suggest certain tumors, such as **hemangiopericytoma** or **solitary fibrous tumors**.
    \item **Giant cells**: Present in **giant cell tumors of bone**, **tenosynovial giant cell tumors**, and **giant cell-rich soft tissue tumors**.
\end{itemize}

\section{Immunocytochemistry}

Immunocytochemistry is an essential tool in the diagnosis of soft tissue tumors. It is employed on most biopsy samples and plays a crucial role in identifying the expression of specific proteins within tissues, aiding in the determination of cell origin and differentiation. In this section, we will discuss key principles of immunocytochemistry and relevant markers used in soft tissue tumor diagnostics.

\subsection{General Principles of Immunocytochemistry}
Immunocytochemistry is akin to extracting biological data from tissues, particularly protein data, which provides information crucial for diagnosis. To draw an analogy, immunocytochemistry has both a “hardware” and a “software” component. The hardware is the actual process and equipment used in extracting the data, while the software is the interpretation of the results. The same immunocytochemistry test can yield different conclusions depending on interpretation, highlighting the need for careful and informed analysis.

There are two fundamental aspects to keep in mind when interpreting immunocytochemistry:

\begin{enumerate}
    \item \textbf{Internal Positive Control:} A negative immunocytochemistry result, in the absence of a reliable internal positive control, must be approached cautiously. This is particularly important in soft tissue tumors, where many antibodies do not have internal positive controls. For example, in solitary fibrous tumors (SFT), STAT6 immunocytochemistry is often used. However, as there is no internal control for STAT6, a negative result does not definitively rule out SFT. Without proper controls, there is a risk of technical artifacts leading to false conclusions.
    
    \item \textbf{External Control:} When internal controls are absent, external controls, such as tissue known to express the antigen in question, can be used. While not ideal, this provides a baseline for comparison. For instance, a small piece of a known solitary fibrous tumor can serve as an external control for STAT6 immunocytochemistry. However, it is important to note that the fixation time and handling of the external control tissue may differ from the biopsy sample, leading to variability in staining results. This can be particularly problematic when dealing with consultations from external laboratories, which may have different fixation protocols.
\end{enumerate}

\noindent
In summary, internal controls are the gold standard, followed by external controls. However, in all cases, interpretation must consider the biology of the tissue and technical limitations.

\subsection{Intermediate Filaments: Keratins}
Intermediate filaments, particularly keratins, are widely used in soft tissue tumor diagnostics. Keratins are categorized by molecular weight and acidity, following a classification from an early work by Moll et al. In soft tissue tumors, keratin expression often signals epithelial differentiation. The most important examples include:

\begin{itemize}
    \item \textbf{Synovial Sarcoma:} Exhibits biphasic growth with true glands, expressing keratin and morphologically appearing as epithelial cells.
    \item \textbf{Epithelioid Sarcoma:} Another sarcoma type expressing keratin, making it important in differential diagnosis.
    \item \textbf{Other Tumors:} Smooth muscle tumors (keratin-positive in about 30\% of cases), melanoma, epithelioid angiosarcoma, and Ewing's sarcoma can also display keratin positivity.
\end{itemize}

Keratins in sarcomas, though rare, can occur. The presence of extensive keratin positivity, however, should always raise suspicion of a possible metastasis or sarcomatoid carcinoma. Therefore, if the tumor is keratin-positive and does not fit with the typical diagnoses mentioned above (e.g., synovial sarcoma, epithelioid sarcoma, leiomyosarcoma, melanoma, angiosarcoma, Ewing sarcoma), one must first consider the possibility of an epithelial tumor. Clinically, ruling out carcinoma or metastasis is essential in such cases.

\subsection{CD34}
CD34 is another important marker in soft tissue tumors. It is rarely, if ever, expressed in carcinomas, making it useful in distinguishing sarcomas from epithelial tumors. For example, pairing keratins and CD34 in immunocytochemistry can help differentiate between epithelial neoplasms and non-epithelial soft tissue tumors.

\subsection{Mucins (EMA)}
The epithelial membrane antigen (EMA, also known as MAC-1) is commonly used to identify epithelial cells in tumors. Many epithelial cancers are positive for EMA, making it a useful marker. It also marks a subset of sarcomas, such as synovial sarcoma, and can be positive in perineurial cells and plasma cells. EMA has broad applications in pathology, allowing for the identification of different cell types in complex tumors.

\subsection{Smooth Muscle Markers}
Smooth muscle tumors require a combination of morphological assessment and immunocytochemical markers for diagnosis. Key markers include:

\begin{itemize}
    \item \textbf{Smooth Muscle Actin (SMA):} This marker, particularly clone 1A4, is sensitive but can also stain other spindle cells. It helps identify smooth muscle differentiation but is not specific to smooth muscle.
    \item \textbf{Muscle-Specific Actin (MSA):} Clone HHF35 highlights both smooth and skeletal muscle. It also marks myofibroblastic differentiation.
    \item \textbf{Desmin:} Desmin is more specific for smooth and skeletal muscle and helps differentiate between muscle tumor types.
    \item \textbf{Caldesmon (H-Caldesmon):} Highly specific for smooth muscle differentiation.
    \item \textbf{Myogenin and MyoD1:} These are specific markers for skeletal muscle differentiation. Myogenin is very specific but less sensitive, as it only marks a small portion of the myogenesis process. MyoD1 generally shows more diffuse positivity in skeletal muscle tumors.
\end{itemize}

\subsection{Peripheral Nerve Sheath Tumors (PNST)}
Markers for peripheral nerve sheath tumors include:

\begin{itemize}
    \item \textbf{S100 Protein:} S100 is a versatile marker that stains a wide range of cells, including Schwann cells, chondrocytes, adipocytes, and myoepithelial cells. It is frequently used to identify peripheral nerve sheath tumors.
    \item \textbf{SOX10:} A marker often used in conjunction with S100 to help identify peripheral nerve sheath tumors. Unlike S100, SOX10 does not stain chondrocytes or adipocytes, providing further specificity.
    \item \textbf{GLUT1 and EMA:} GLUT1 and EMA can be used to identify perineurial cells. GLUT1 also stains red blood cells, so care must be taken in interpreting its positivity within vessels.
\end{itemize}

\subsection{Staining for Molecular Events}
Some immunocytochemical markers are proxies for specific molecular alterations. These markers allow for a more targeted approach to diagnosis, particularly when molecular testing is not immediately available. In the next message, we will list specific markers used as surrogates for molecular testing in soft tissue tumor diagnostics.


